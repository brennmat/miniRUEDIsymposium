\begin{conf-abstract}
{Gases as artificial tracers to study SW-GW interactions}
{Théo Blanc}
{University of Neuchatel, Switzerland, and Eawag, Swiss Federal Institute of Aquatic Science and Technology}
{Understanding the interaction between groundwater (GW) and surface water (SW) is crucial for effective drinking water management and the well-being of ecosystems. However, comprehending these interactions can be quite challenging due to the subsurface heterogeneity, which causes preferential groundwater flow.
Tracers are commonly used to study GW-SW interactions. Traditional artificial tracers like dyes are difficult to handle. Moreover, the coloring of rivers may cause negative public perception. Recent advances in portable mass spectrometry facilitate their direct and continuous measurement in the field [1], enabling their operational use in GW [2]. Nevertheless, there are technical and economic barriers that hinder the routine application of gas tracers for studying SW-GW interactions, particularly the substantial volumes of gas (especially noble gases) required for injection into rivers, which can be cost-prohibitive.

We present a cost-effective method for diffusive gas injection into rivers using easily available materials. We tested our approach with the noble gas helium (He) in a pre-alpine river connected to an alluvial aquifer (Emmental, Switzerland). We present a cost-effective method for diffusive gas injection into rivers using easily available materials. We tested our approach with the noble gas helium (He) in a pre-alpine river connected to an alluvial aquifer (Emmental, Switzerland). Gas injection was sustained for 35 days and oversaturated the river water with He by one order of magnitude compared to natural conditions. Dissolved gas concentrations (He, O2, N2, Ar, and Kr) were monitored in the river, a drinking water well, and several piezometers Gas measurements provided quantitative information on connectivity and river infiltration dynamics. The results demonstrated a direct hydraulic connection between the infiltrating river and the drinking water well. Moreover, results from a pulse gas tracer test, conducted by injecting Krypton directly into the aquifer, highlighted the existence of preferential groundwater flow paths in the aquifer, with measured groundwater velocities above 3 mm/s (13 m/h).

[1] ES\&T, 2016, 50, 13455-13463; ES\&T, 2017, 51, 846-854; [2] Front. Water, 2022, 4, 925294.}
\end{conf-abstract}
