\begin{conf-abstract}
{Exploring the Potential of On-Site Gas Analysis in Aquatic Systems}
{David C.\ Finger}
{Reykjavik University, Iceland}
{The study of aquatic ecosystems plays a pivotal role in understanding and mitigating the impacts of environmental changes on water quality, biodiversity, and ecosystem health. The application of on-site gas analysis techniques provide a more profound insights into the dynamic biogeochemical processes that govern aquatic environments. 
Icelandic hydropower reservoirs are fed by millennial old glacial melt water that has a preindustrial CO2 content. When the carbon dioxide (CO2) concentration in the water of hydropower reservoirs is lower than in the atmosphere, CO2 uptake can occur. The lower CO2 concentration in the reservoir water compared to the atmosphere can lead to the dissolution of CO2 from the air into the water.

Geothermal energy production produces emissions of Carbon Dioxide (CO2), Sulfur Compounds (H2S), Volatile Organic Compounds (VOCs), Hydrogen and Methane (CH4), to name the most important ones. The assessment of emissions and monitoring is essential to reduce GHG emissions and identify the potential for a circular use of these natural emissions [1]. 
On-site gas analysis plays a crucial role in identifying the origin of runoff water from glaciers, snow, and rain by examining the isotopic composition of gases dissolved in the water. The isotopic Signatures can reveal the ratio of snow-, ice and rain runoff, helping hydropower operators manage water resources and adapt to climate change [2].
This presentation will conclude with a call for international research cooperation on on-Site Gas Analysis in Aquatic Systems in Iceland.

1. Finger, D.C., Saevarsdottir, G., Svavarsson, H.G. et al. (2021) Improved Value Generation from Residual Resources in Iceland: the First Step Towards a Circular Economy. Circ.Econ.Sust. 1, 525–543. DOI 10.1007/s43615-021-00010-7\\
2. Finger, D., Hugentobler, A., Huss, M., Voinesco, A., Wernli, H., Fischer, D., Weber, E., Jeannin, P.-Y., Kauzlaric, M., Wirz, A., Vennemann, T., Hüsler, F., Schädler, B., and Weingartner, R. (2013) Identification of glacial meltwater runoff in a karstic environment and its implication for present and future water availability, Hydrol. Earth Syst. Sci., 17, 3261–3277, DOI 10.5194/hess-17-3261-2013}
\end{conf-abstract}
