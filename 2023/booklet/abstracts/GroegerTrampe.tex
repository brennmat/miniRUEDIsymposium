\begin{conf-abstract}
{On-site measurement of excess N2 ``under pressure''}
{Jens Gröger-Trampe}
{State Authority for Mining, Energy and Geology (LBEG), Germany}
{Nitrate pollution of aquifers in several parts of Germany is a matter of rising concern. Recent changes to the German Groundwater Ordinance therefore require all federal German states to provide data to quantify denitrification in relevant aquifers by the end of 2025, additionally to nitrate concentrations. The method of choice is the calculation of excess N2 from N2/Ar-measurements. To address potential analytical issues, the State Authority for Mining, Energy and Geology (LBEG) and the Lower Saxony Water Management, Coastal Defence and Nature Conservation Agency (NLWKN) are regularly hosting the nationwide interlaboratory tests for N2/Ar-measurements of groundwater. To assure data quality and to assess a variety of effects, LBEG developed a quality control tool (N2ArCheck, unreleased beta version).

miniRUEDI measurements are used in this context to address several aspects: The on-site measurement is an independent method for comparison with lab data, helps to get a more precise picture of excess air formation than measurements of solely N2 and Ar and is used to evaluate degassing effects. While measurements at several sites are in good agreement with lab data, tests on other sites show some issues. Some of these on-site measurements reveal major gas losses compared to lab measurements. This effect occurs at groundwater wells with elevated concentrations of dissolved gasses (total dissolved gas pressure (TDGP) up to 1.5 atm). Thus, a series of tests was set up to address this problem and explore the limits of miniRUEDI measurements in groundwater wells with elevated TDGPs.}

\end{conf-abstract}
