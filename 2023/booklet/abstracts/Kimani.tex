\begin{conf-abstract}
{Analyzing Unexpected Gas Seep in Mbeya, Tanzania with a Portable Mass Spectrometer (“MiniRuedi”)}
{Clarah Kimani}
{University of Dar es Salaam, Tanzania}
{A study was conducted in Mbeya region Tanzania, to investigate unexpected gas seepage from a dry well at Harrison Uwata Girls Secondary School. Gas samples were collected and analyzed using a portable mass spectrometer (“MiniRuedi”), an analytical instrument to determine their composition. The study focused on the Rungwe Volcanic Province (RVP), situated within the East African Rift System, known for active magmatic activity, including vigorous gas and volatiles emanations. The primary gas in the seep was found to be carbon dioxide (CO2), with trace amounts of nitrogen (N2), argon (Ar), and helium (He). The analysis traced the sources of these gases to contributions from both the Earth's atmosphere and the subsurface, particularly the Earth's mantle.
The concentration of CO2 in the seep was found to vary based on proximity to volcanic centers, with closer locations having higher CO2 concentrations ($\sim$90\%). This decrease in CO2 concentration with distance was attributed to dilution by other gases from sources like the atmosphere and the Earth's crust. The study emphasizes the importance of understanding the origin of CO2 in these seeps and calls for further research to explore its nature, storage, and potential economic extraction and utilization as a resource.

The investigation at Harrison Uwata Girls Secondary School gas seep in the RVP has significant implications for understanding the geology, geochemistry, and geological processes in the area and the broader East African Rift System, offering opportunities for future research and resource exploration. Additionally, the study highlights the presence of different hydrothermal systems in the RVP, including cold-gassy and hot-gaseous systems, with varying characteristics and potential hazards.}
\end{conf-abstract}
