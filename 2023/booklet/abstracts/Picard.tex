\begin{conf-abstract}
{Analyse of discrete (ground)water dissolved gases samples using the miniRUEDI: The tricks of a simplified protocol and tests for passive gas samplers}
{Antoine Picard, Matéo Lacheux, Christin Müller, Florent Barbecot, and José Corcho}
{UQAM-Geotop, Montreal, Canada}
{While the miniRUEDI is widely used for continuous measurements, some projects in our laboratory imply to measure gases in discrete samples. To test the feasibility of such measurements, an in-house made system was built to connect the miniRUEDI to stainless steel tubes of determined volumes. Numerous protocols and tests have been performed to enhance a methodology for the analysis of such finite-volume samples such as long-term analyses, short-term analyses, and signal integration as well as diffusion effects within the miniRUEDI line system, filament lightning duration and gas leaks testing. The final protocol allows for a precise simultaneous determination of gases concentration (typically: He, N2, O2, Ar and Kr) within two hours of analysis. This new discrete measurement protocol has been tested on gas tracing experiment and for passive samplers equilibrated with atmosphere and surface water. The results are very promising as the practical implication of this work is to give a simple access to discrete sampling analyses of noble gases. Then, scientific projects such as groundwater influx quantification to rivers, investigation of vertical stratification in boreholes/aquifers/lakes, mapping of artificial tracing experiments, excess air, degassing and groundwater dating will find a great support from the miniRUEDI community.}
\end{conf-abstract}
