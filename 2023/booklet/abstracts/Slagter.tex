\begin{conf-abstract}
{Relative partial pressures in seawater of nitrogen, oxygen and argon; MiniRUEDI measurements in support of in-situ open ocean primary production and CO\textsubscript{2} flux determinations}
{Hans A. Slagter\textsuperscript{1}, Maria Ll. Calleja\textsuperscript{1,2}, Hedy M. Aardema\textsuperscript{1,3}, Ralf Schiebel\textsuperscript{1}, Antonis Dragoneas\textsuperscript{1}, Lena Heins\textsuperscript{1}, Isabella Hrabe de Angelis\textsuperscript{1}, Gerald Haug\textsuperscript{1,3}}
{1 Max Planck Institute for Chemistry Mainz Germany, 2 University of the Balearic Islands Palma de Mallorca Spain, 3
ETH Zurich Switzerland}
{Marine primary production and microbial respiration are cornerstone global processes and integral to our understanding of climate and biogeochemical cycling. Despite long-time recognition, there remains a paucity of high temporal and spatial resolution data for both the elucidation and decoupling of seasonal and latitudinal variability. Modern approaches such as the collection of high-throughput data from miniaturised instrumentation allow for applications of such measurements in novel platforms. To this end, net community production, that is the debit of gross production and respiration, is derived from and studied in relation to a multitude of properties measured on the blue water research sailing yacht Eugen Seibold. The ratio between oxygen and argon in particular, derived from a miniRUEDI GE-MIMS and set-up in combination with pCO2 data and microbial community standing stock information, helps to inform carbon cycling. In addition, nitrogen cycling processes may be further informed by the same set-up. The miniRUEDI instrument on S/Y Eugen Seibold measures from a GE membrane integrated in our FerryBox system, which continually samples from a keel inlet at a depth of 3.2 m. Here we present our implementation and preliminary results along a North-South transect in the Atlantic Ocean, stretching from the polar circle to the equator roughly along the 20° W meridian, collected during expeditions in 2020 and 2021. This transect crosses from the mesotrophic high North Atlantic into the oligotrophic subtropical Atlantic, crossing several areas of nutrient upwelling. Transects crossing the Atlantic into the Pacific extend this data going forward.
}
\end{conf-abstract}
