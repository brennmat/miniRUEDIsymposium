\begin{conf-abstract}
{miniRUEDI as handy tool for onsite monitoring of gas injection tests in the frame of GAST experiment}
{Yama Tomonaga$^1$, Emiliano Stopelli$^2$, Jocelyn Gisiger$^3$, Bill Lanyon $^4$and Thomas Spillmann$^2$}
{1: Entracers GmbH, Switzerland, 2: Nagra, Switzerland, 3: Solexperts AG, Switzerland, 4: Fracture Systems Ltd, United Kingdom}
{In radioactive waste repositories gas generation is expected from the degradation of organic substances and metal components. To manage gas generation while ensuring the containment of radionuclides, concepts for gas-permeable plugs and seals have been developed.

The Gas permeable Seal Test (GAST) at Grimsel Test Site is an international project (ANDRA, NAGRA, NWMO, NWS) aimed at demonstrating the feasibility and functionality of a gas-permeable seal made of a sand/bentonite mixture, at 1:1 scale and realistic boundary conditions. After progressive seal saturation, gas injection tests were conducted between May 2022 and August 2023, using noble gases as tracers of gas transport through the seal section of the experiment.
The miniRUEDI portable mass spectrometer system has been shown to be a very versatile and reliable instrument for the mid- to long-term assessment of gas dynamics both in engineered and natural environments. Thus, it was deployed onsite at the GAST experiment to:
\begin{itemize}
\item detect any potential gas leaks from the experimental set-up
\item qualitatively monitor the changes in gas composition at the outflow line of the experiment, via coupling with a semi-permeable membrane module
\item allow real-time support for the operational decisions throughout all gas injections phases 
\item be compared for quality check with off-site analyses (e.g., to infer equipment biases)
\end{itemize}
In this contribution we present and discuss observations related to the first phase of GAST experiment using a 2\% He-spiked N$_2$ gas.}
\end{conf-abstract}



