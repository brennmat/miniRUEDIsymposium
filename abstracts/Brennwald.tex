\begin{conf-abstract}
{River rafting with the miniRUEDI}
{Connor P. Newman$\textsuperscript{1}$, Eric C Humphrey$\textsuperscript{1}$, Matthias S Brennwald$\textsuperscript{2}$, W. Payton Gardner$\textsuperscript{3}$, Kelli M Palko$\textsuperscript{1}$ and Michael N Gooseff$\textsuperscript{4}$}
{1: U.S. Geological Survey, 2: Eawag, Swiss Federal Institute of Aquatic Science and Technology, 3: University of Montana USA, 4: University of Colorado USA}
{Saline geothermal systems in the western United States are major sources of solutes that negatively affect downstream water use. Discharge from these systems is commonly manifested as both discrete springs and diffuse outfow along the riverbed. Discrete discharges may be quantified using physical measurements, but large diffuse discharges are difficult to quantify. To quantify total discharge from several large saline geothermal systems in the Upper Colorado River Basin high-resolution noble gas measurements were made from a boat-mounted portable gas equilibrium membrane-inlet mass spectrometer (miniRUEDI) in two locations in the western United States: the Colorado River near Glenwood Canyon and the Virgin River near Pah Tempe springs. Geothermal discharge in both locations has distinctive noble gas signatures, with He concentrations in springs enriched three orders of magnitude above atmospheric equilibrium (about $10^{-5}$\,ccSTP/g). We hypothesize that the enriched He concentrations of discrete (springs) and diffuse (riverbed) discharge from these systems will have an influence on the noble gas composition of river water, allowing for the total geothermal discharge in each location to be quantified by helium mass balance. Continuous noble gas measurements also allow for direct estimation of the air/water He gas transfer rate), which commonly requires time-consuming gas injections. Results from the Glenwood Canyon area indicate that He concentrations in the river undergo an order of magnitude increase (to about $10^{-7}$\,ccSTP/g). Elevated helium concentrations in the river extend kilometers downstream from mapped inflows and indicate substantial diffuse inflows or slow degassing of geothermal He. Mass balance modeling accounting for helium inflow and degassing indicates a total geothermal discharge of 425 to 850\,L/s, compared to a previous estimate of 300\,L/s from discrete springs. Results from the Pah Tempe area indicate He concentration in the river enriched two orders of magnitude above atmospheric equilibrium (up to about $10^{-6}$\,ccSTP/g). Geothermal discharge to the Virgin River in this reach is estimated at approximately 250\,L/s, similar to the flux measured by differential gaging. Data allow for estimation of the He gas transfer rate in the Colorado River (40 m/d) and Virgin River (80 m/d), illustrating the utility of the high-resolution measurements in quantifying this important parameter.}
\end{conf-abstract}
