\begin{conf-abstract}
{Tracing and quantifying microbes in riverbank filtration sites combining online flow cytometry and noble gas analysis}
{Friederike Currle}
{University of Basel}
{Understanding microbial transport in surface water – groundwater systems is crucial for drinking water management. Particularly in the context of climate change, the quality of groundwater pumped near streams might be affected by high microbial loads after heavy rain, peak flow and spring snowmelt events. Dissolved noble gases have been shown to be conservative tracers and provide information on pathways and travel times of groundwater. Although it is known that due to size exclusion, microbes appear to travel faster than solutes, most hydrological tracer methods target groundwater movement and solute transport, while specific tracers for microbial transport are not yet considered for protection zone delineation of drinking water supply wells. Recently, online flow cytometry (FCM) has been shown to be a promising tool to track on site, continuously and in near-real time the movement of microbes in riverbank filtration settings (Besmer et al., 2016). Beyond direct cell counting, advanced computational tools enable to extract automatically relevant features from the multivariate FCM data describing the phenotypic diversity of the microbial community.

Aiming to understand microbial transport behavior in surface water – groundwater systems and develop tracer methods to track their movement, we combined online FCM with online (noble) gas analysis at a riverbank filtration site in the Emme valley, Switzerland (Schilling et al., 2022). Dissolved gas concentrations and microbial community patterns (measured using the gas equilibrium-membrane inlet portable mass spectrometer miniRUEDI (Brennwald et al. (2016), Gasometrix GmbH), the electronic radon detector Rad7 (DURRIDGE), and the online flow cytometer BactoSense (bNovate Technologies SA)) were monitored continuously over a period of several months of river restoration activity inside the river, a piezometer next to the river, and nearby riverbank filtration wells. Systematic changes in the microbial and dissolved gas patterns could be observed in reaction to a 2-year peak flow event, river restoration activities, and spring snowmelt events.

In summary, this combination of state-of-the-art analytical techniques allows to track and quantify microbial pathways from surface water into and through an alluvial aquifer. Furthermore, the setup increases understanding of reactive microbial transport compared to the transport of conservative dissolved gases and, highlights the potential of environmental DNA as a hydrological tracer technique.}
\end{conf-abstract}
