\begin{conf-abstract}
{Gas monitoring while Drilling the Ivrea-Verbano zonE (DIVE) with miniRUEDI}
{Hugo Dutoit}
{ISTerre, France}
{MiniRUEDI is a versatile QMS allowing dissolved gas analysis with quick results directly on the field. However, with proper calibrations this mass spectrometer also enables in situ free gas analysis. In order to experiment long time running free gases measurement with the miniRUEDI, the latter was implemented to an On-Line Gas Analysis set up (OLGA) for mud-gas monitoring while drilling. The ICDP project "Drilling the Ivrea Verbano zonE (DIVE)" explores the Ivrea Verbano Zone in the Southern Alps of Italy, the probably most complete pre-Permian lower crust–upper mantle transition worldwide, by deep scientific drilling. A first borehole has been completed near the city of Ornavasso in mid-December 2022, reaching a final depth of 578.5 m, with excellent drill core recovery (100\%). The drilling was accompanied by various scientific experiments, including the continuous extraction, measurement and sampling of gases from the circulating drilling fluid (OLGA). The gas phase was continuously measured with two quadrupole gas mass spectrometers (miniRUEDI © and Pfeiffer Omnistar ©) for Ar, H2, He, N2, O2, CH4 and CO2, a gas chromatograph for hydrocarbons (CH4, C2H6, C3H8 and i/n C4H10), and a radon detector for 222Rn. Initial results show a correlation between formation gases in drilling mud and the drilled fault and fracture zones. In addition to the unavoidable input of atmospheric gases in drilling mud, the most non-atmospheric gases extracted from drilling mud are hydrogen (up to ~1.2 vol.-\%) and methane (up to ~0.3 vol.-\%). Likewise, helium content was sometimes found to be higher than atmospheric. MiniRUEDI appeared to be a very efficient and robust tool facing the many issues occurring during the drilling period. Moreover, the obtained results were in good agreement with the initial OLGA set-up.}
\end{conf-abstract}
