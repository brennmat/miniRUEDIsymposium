\begin{conf-abstract}
{Development of an Equilibrator-Inlet Mass Spectrometer (EIMS) for continuous N2, O2 and Ar measurements to quantify nitrogen fixation in the Baltic Sea}
{Sören Iwe}
{Leibniz-Institute for Baltic Sea research (IOW) Warnemünde, Germany}
{The contribution of nitrogen fixation by cyanobacteria to the N budget of the Baltic Sea equals the total sum of DIN input from riverine and airborne sources, varying between 300 kt-N/yr and 800 kt-N/yr. The vast range is due to interannual fluctuations and significant uncertainties in the various techniques used to determine N2 fixation and in extrapolating local study to entire basins. To overcome some of the limitations, we introduce a new approach based on large-scale records of surface water N2 depletion during summer. 
To determine the concentration of N2 in surface water, a membrane contactor (Liquicel) is utilized to establish gas phase equilibrium for atmospheric gases dissolved in seawater. The mole fractions for N2, O2 and Ar in the gas phase are determined by mass spectrometry and yield the concentration of these gases by multiplication with the total pressure and the respective solubility constants. 
After thorough laboratory tests concerning the precision and accuracy, the measurement system, an Equilibrator-Inlet Mass Spectrometer (EIMS), was deployed on a voluntary observing ship (VOS, “Finnmaid”) during June/July of summer 2023. By conducting repeated transects 2-3 times per week over a distance of approximately 1000 km between the Mecklenburg Bight and the Gulf of Finland, the results offer a high-resolution time series of N2 concentration changes induced by nitrogen fixation. Additionally, Ar measurements are used to account for the air-sea N2 gas exchange. Furthermore, the biological oxygen saturation $\Delta$O2/Ar can be utilized to characterize the production phase. 
In connection with these measurements, further investigations of nitrogen fixation and its vertical distribution were undertaken during a research cruise with a similar measurement setup.
The initial results appear promising, though a final analysis and evaluation of the data is still pending. Our objectives are to identify various factors that initiate and potentially limit the growth of cyanobacteria.}
\end{conf-abstract}
