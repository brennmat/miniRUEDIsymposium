\begin{conf-abstract}
{Development of an Equilibrator-Inlet Mass Spectrometer (EIMS) for continuous N2, O2 and Ar measurements to quantify nitrogen fixation in the Baltic Sea}
{Sören Iwe}
{Leibniz-Institute for Baltic Sea research (IOW) Warnemünde, Germany}
{Nitrogen fixation by cyanobacteria is a common phenomenon in the Baltic Sea. Its contribution to the N budget is of the same order of magnitude as the combined sum of riverine and airborne DIN input, varying between 300 kt-N/yr and 800 kt-N/yr. The vast range is due to interannual fluctuations and significant uncertainties in the various techniques used to determine N2 fixation and in extrapolating local study to entire basins. To overcome some of the limitations, we introduce a new approach based on large-scale records of surface water N2 depletion during summer, when the probability of a cyanobacteria bloom is high. Additionally, Ar measurements are performed to account for the air-sea N2 gas exchange. Furthermore, the biological oxygen saturation $\Delta$O2/Ar can be utilized to further characterize the production phase. 
For our studies, we developed an Equilibrator-Inlet Mass Spectrometer (EIMS) designed for deployment on a ferry line, enabling repeated transects along the same route and providing high temporal and spatial resolution data for N2, O2 and Ar gas concentrations in the surface water. During June/July of summer 2023, we performed surface water measurements on a voluntary observing ship (VOS, “Finnmaid”) which is travelling 2-3 times per week over a distance of about 1000 km between the Mecklenburg Bight and the Gulf of Finland. In connection with these measurements, further investigations of nitrogen fixation and its vertical distribution were undertaken during a research cruise in order to identify the processes that control nitrogen fixation and the linked biomass production.
The initial results appear promising, though a final analysis and evaluation of the data is still pending. 
In the future, we plan to use the 2023 dataset to establish a nitrogen budget for the Baltic Proper. Our objectives are to identify various factors that initiate and potentially limit the growth of cyanobacteria.}
\end{conf-abstract}
