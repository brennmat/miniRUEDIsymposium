\begin{conf-abstract}
{Application of noble gases to urnaium deposite sites and net-zero}
{Ye Ji Kim}
{University, Republic of Korea}
{Isolation of spent nuclear fuel (SNF) disposed underground from people and the environment is crucial. Even in the cases where the nuclide species escape through the engineering barriers, these have to be slowly transported through the natural barrier for over 10 million years until they decay to low hazardous materials. Hence, understanding whether geological structure can adequately control the movement of nuclides and what changes occur in the migration processes are crucial. These kinds of long-term observations and analyses can be conducted by analyzing cases that have been in progress for a long time. Therefore, this project aims to establish the basis for natural analogue research by securing a uranium mineral research site in Korea. Along with this project goal, one of our goals is to characterize the transport of uranium (U) and its daughter product, helium (He), within the groundwater of our study site, Boeun. Samples from the dry season and wet season were each collected with copper tubes and the miniREUDI respectively. Dry season samples showed radiogenic 4He at a depth of 80 m, which coincided with the natural gamma ray peak at this depth. Whereas for the wet season samples, while argon (Ar) showed similar trends to the dry season, He showed hardly any variation between the wells. Summer samples were also characteristic of varying tritium values, while carbon-14 values were consistent. In addition to the aforementioned individual sampling campaigns, an event of continuous extraction of groundwater with the analyzation of helium and other parameters (O2, CO2, CH4) that are related to the mobilization of uranium, is anticipated to provide a wider understanding of the transport relation between uranium and helium.}
\end{conf-abstract}
