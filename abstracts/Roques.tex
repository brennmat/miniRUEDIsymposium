\begin{conf-abstract}
{In situ observation of helium and argon release during fluid-pressure-triggered rock deformation}
{Clément Roques}
{University of Neuchatel, Switzerland}
{Temporal changes in groundwater chemistry can reveal information about the evolution of flow path connectivity during crustal deformation. Here, we report transient helium and argon concentration anomalies monitored during a series of hydraulic reservoir stimulation experiments measured with an in situ gas equilibrium membrane inlet mass spectrometer. Geodetic and seismic analyses revealed that the applied stimulation treatments led to the formation of new fractures (hydraulic fracturing) and the reactivation of natural fractures (hydraulic shearing), both of which remobilized (He, Ar)-enriched fluids trapped in the rock mass. Our results demonstrate that integrating geochemical information with geodetic and seismic data provides critical insights to understanding dynamic changes in fracture network connectivity during reservoir stimulation. The results of this study also shed light on the linkages between fluid migration, rock deformation and seismicity at the decameter scale.}
\end{conf-abstract}
