\begin{conf-abstract}
{Quantifying Groundwater Recharge Dynamics and Unsaturated Zone Processes in Snow-Dominated Catchments via On-Site Dissolved (Noble) Gas Analysis}
{Schilling, O.S.\textsuperscript{1,2,3}, Parajuli, A.\textsuperscript{3}, Tremblay Otis, C.\textsuperscript{4}, Müller, T.U.\textsuperscript{3}, Antolinez Quijano, W.\textsuperscript{3}, Tremblay Y.\textsuperscript{3}, Brennwald, M.S.\textsuperscript{2}, Nadeau D.F.\textsuperscript{3}, Jutras, S.\textsuperscript{3}, Kipfer, R.\textsuperscript{2}, Therrien, R.\textsuperscript{3}}
{1 University of Basel, Switzerland, 2 Swiss Federal Institute of Aquatic Science and Technology (Eawag), 3 Universite Laval, Canada, 4 Universite de Neuchatel, Switzerland}
{Snowmelt contributes a significant fraction of groundwater recharge in snow-dominated regions, making its accurate quantification crucial for sustainable water resources management. While several components of the hydrological cycle can be measured directly, catchment-scale recharge can only be quantified indirectly. Stable water isotopes are often used as tracers to estimate snowmelt recharge, even though estimates based on stable water isotopes are biased due to the large variations of $\delta$\textsuperscript{2}H and $\delta$\textsuperscript{18}O in snow and the difficulty to measure snowmelt directly. To overcome this gap, a new tracer method based on on-site measurements of dissolved He, \textsuperscript{40}Ar, \textsuperscript{84}Kr, N\textsubscript{2}, O\textsubscript{2}, and CO\textsubscript{2} is presented. The new method was developed alongside classical tracer methods (stable water isotopes, \textsuperscript{222}Rn, \textsuperscript{3}H/\textsuperscript{3}He) in a highly instrumented boreal catchment. By revealing (noble gas) recharge temperatures and excess air, dissolved gases allow (i) the contribution of snowmelt to recharge, (ii) the temporal recharge dynamics, and (iii) the primary recharge pathways to be identified. In contrast to stable water isotopes, which produced highly inconsistent snowmelt recharge estimates for the experimental catchment, dissolved gases produced consistent estimates even when the temperature of snowmelt during recharge was not precisely known. As dissolved gases are not controlled by the same processes as stable water isotopes, they are not prone to the same biases and represent a highly complementary tracer method for the quantification of snowmelt recharge dynamics in snow-dominated regions. Furthermore, an observed systematic depletion of N\textsubscript{2} in groundwater provides new evidence for the pathways of biological N-fixation in boreal forest soils.}
\end{conf-abstract}
